\documentclass{article}
\usepackage{ulem}
\usepackage{graphicx} % Required for inserting images

\title{Font Style}
\author{myTest Account}
\date{July 2023}

\begin{document}
\maketitle

\section{Introduction}

If we want to underline a word or a sentence, we use the command \underline{underline}. On the other hand we might want to use bold fonts, and we do that by invoking the command \textbf{textbf}. Finally, the other very common way to put emphasis on a text is the use of italics, and for that we use the command \textit{textit}. \emph{emphasis} is also italics (depending on the documentclass), but not always.

\uline{This command} is equivalent to the command underline. Ulem changes the functionality of emph to underline. \uuline{This other command} might add something new to our text. A wave underline would look like this: \uwave{this is an example}. \sout{This content is to be removed}. \xout{This is censored}. In order to have a dashed underline, we use the command \dashuline{dashuline}. For the dotted underline, we use the command \dotuline{dotuline}.

\end{document}